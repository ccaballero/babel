\documentclass[11pt]{beamer} 
\usepackage[spanish]{babel}
\usepackage[utf8]{inputenc}

\usepackage{lmodern}
\usepackage[T1]{fontenc}
\usepackage{textcomp}

\usetheme{Copenhagen}
\usecolortheme{seagull}

\usepackage{tikz}
\usetikzlibrary{mindmap,trees}
\usepackage{tikz-er2}

\title{\bf Babel}
\author{Carlos Caballero}
\institute{scesi}
\date{2011/10/24}

\begin{document}

\begin{frame}
\hspace*{0.22cm}
\includegraphics[width=0.01\textwidth]{img/babel.png}
\includegraphics[width=0.02\textwidth]{img/babel.png}
\includegraphics[width=0.04\textwidth]{img/babel.png}
\includegraphics[width=0.08\textwidth]{img/babel.png}
\includegraphics[width=0.16\textwidth]{img/babel.png}
\includegraphics[width=0.32\textwidth]{img/babel.png}
\includegraphics[width=0.16\textwidth]{img/babel.png}
\includegraphics[width=0.08\textwidth]{img/babel.png}
\includegraphics[width=0.04\textwidth]{img/babel.png}
\includegraphics[width=0.02\textwidth]{img/babel.png}
\includegraphics[width=0.01\textwidth]{img/babel.png}
\titlepage
\end{frame}

\section{Definición}
\begin{frame}
Babel es la primera pieza de un proyecto destinado a minimizar las dificultades
que generalmente se sufren en el contexto en el que nos desempeñamos.
\end{frame}

\subsection{Los problemas}
\begin{frame}
\begin{center}
\begin{tikzpicture}
\path[mindmap,concept color=black,text=white]
node[concept]{\bf Problemas de accesibilidad}[clockwise from=-60]
child[concept]{node[concept]{\bf Problemas del co\-no\-ci\-mi\-en\-to necesario}[clockwise from=0]
child[concept]{node[concept]{\bf Problemas del ti\-em\-po requerido}[clockwise from=90]
child[concept]{node[concept]{\bf Problemas de com\-pro\-mi\-so}[clockwise from=-180]
child[concept]{node[concept]{\bf Ob\-je\-ti\-vo}}}}}
\end{tikzpicture}
\end{center}
\end{frame}

\subsection{Funciones}
\begin{frame}
Babel es un sistema web que ofrece acceso a una colección de documentos en formato pdf.\\ \pause
Posee las siguientes caracteristicas: \pause
\begin{itemize}
\item Los documentos son compartidos por sus usuarios. \pause
\item Estos son revisados, clasificados y publicados por sus administradores. \pause
\item Cualquier persona puede descargar, catalogar y valorar los documentos de manera anonima.
\end{itemize}
\end{frame}

\subsection{Premisas}
\begin{frame}
Este sistema se construyó siguiendo las premisas citadas a continuación:\\ \pause
\begin{itemize}
\item Maximizar el anonimato de todos sus usuarios. \pause
\item Dinamizar el conjunto de posibilidades de clasificación. \pause
\item Ofrecer amplias formas de automatización de tareas comunes.
\end{itemize}
\end{frame}

\section{Descripción}
\begin{frame}[fragile]
\begin{verbatim}
mysql> show tables;
+------------------------+
| Tables_in_babel        |
+------------------------+
| babel_books_catalogs   |
| babel_books_collection |
| babel_books_meta       |
| babel_books_stats      |
| babel_catalogs         |
| babel_catalogs_stats   |
| babel_search_keywords  |
| babel_users            |
+------------------------+
8 rows in set (0.00 sec)
\end{verbatim}
\end{frame}

\begin{frame}
\usetikzlibrary{positioning}
\tikzstyle{every entity} = [top color=white, bottom color=blue!30, draw=blue!50!black!100, drop shadow]
\tikzstyle{every relationship} = [top color=white, bottom color=red!20, draw=red!50!black!100, drop shadow]
\centering
\scalebox{.87}{
\begin{tikzpicture}[node distance=1.5cm, every edge/.style={link}]
\node[entity] (meta) {Meta};
\node[relationship] (have) [left=2.0cm of meta] {has} edge (meta);
\node[entity] (collection) [left=2.0cm of have] {Collection} edge (have);
\node[entity] (catalog) [below=of collection] {Catalogs};
\node[relationship] (composes) [right=2.0cm of catalog] {composes} edge (catalog);
\end{tikzpicture}
}

\end{frame}

\begin{frame}[fragile]
\begin{verbatim}
mysql> describe babel_books_collection;
+------------+------------------+------+-----+
| Field      | Type             | Null | Key |
+------------+------------------+------+-----+
| hash       | char(32)         | NO   | PRI |
| size       | int(10) unsigned | NO   |     |
| directory  | varchar(2048)    | NO   | MUL |
| file       | varchar(2048)    | NO   |     |
| published  | tinyint(1)       | NO   |     |
| tsregister | int(10) unsigned | NO   |     |
| tsupdated  | int(10) unsigned | NO   |     |
+------------+------------------+------+-----+
7 rows in set (0.00 sec)
\end{verbatim}
\end{frame}

\end{document}
