\documentclass[letter,12pt]{article}

\usepackage[T1]{fontenc}
\usepackage{lmodern}
\usepackage{textcomp}
\renewcommand*\familydefault{\sfdefault}

\usepackage[spanish]{babel}
\usepackage[utf8x]{inputenc}

\usepackage[pdftex]{graphicx}
\usepackage{pifont}
\usepackage[
pdfauthor={Carlos Caballero - Ubaldino Zurita},%
pdftitle={Proyecto Babel},%
colorlinks,%
citecolor=black,%
filecolor=black,%
linkcolor=black,%
urlcolor=black
pdftex]{hyperref}

\usepackage{fancyhdr}
\usepackage{lastpage}
\pagestyle{fancy}

% Para la primera página
\fancypagestyle{plain}{
\fancyhead[l]{}
\fancyhead[r]{}
\fancyhead[c]{}
\renewcommand{\headrulewidth}{0.5pt}
\fancyfoot[l]{SCESI \\ Sociedad Científica de Estudiantes de Sistemas e Informática}
\fancyfoot[c]{}
\fancyfoot[r]{\thepage/\pageref{LastPage}}
\renewcommand{\footrulewidth}{0.5pt}}

% Para el resto de páginas
\lhead{Proyecto Babel}
\chead{}
\rhead{\includegraphics[width=0.1\textwidth]{img/scesi.png}}
\renewcommand{\headrulewidth}{0.4pt}
\lfoot{SCESI \\ Sociedad Científica de Estudiantes de Sistemas e Informática \\ 
\url {http://scesi.fcyt.umss.edu.bo}}
\cfoot{}
\rfoot{\thepage/\pageref{LastPage}}
\renewcommand{\footrulewidth}{0.4pt}

\title{\bf Babel}
\author{Carlos Caballero\\Ubaldino Zurita} 

\begin{document}
\maketitle
\begin{center}\includegraphics[width=0.48\textwidth]{img/babel.png}\end{center}
\begin{center}\url {http://scesi.fcyt.umss.edu.bo}\end{center}
\pagebreak

\tableofcontents
\pagebreak

\begin{abstract}
Babel es una de las piezas de software relacionadas a un gran proyecto que
intenta reducir las brechas de accesibilidad que se han percibido entre la
comunidad estudiantil. En esencia consiste en un sitio web desarrollado en
el lenguaje de programacion php, donde los usuarios pueden compartir, ordenar,
clasificar, catalogar y valorar archivos en formato pdf.

Babel esta concebido con un logica p2p, es decir, esta diseñado pensando en
crear conexiones con otras instancias, ya sea publicas o privadas, de modo
que el rango de busqueda pueda propagarse a una variedad aun mayor que la de
una sola instancia.

Babel ademas es software libre, lo que implica que cualquier usuario puede
instalarse una instancia en su computador y conectarse con otras instancias
accesibles via web.
\end{abstract}
\pagebreak

\section{Introducción}
Debido a que no contamos con una biblioteca virtual en los predios de la universidad se nos dio la idea a realizar un
proyecto llamado {\bf BABEL} que se trata de un centro de información electrónico que proporcionara documentos a sus
usuarios, entendiendo por documento un término muy amplio que casi puede definirse como cualquier tipo de información
fijada en cualquier tipo de soporte digital.
\section{Antecedentes}

\section{Definición del Problema}
La biblioteca tradicional se entiende como una colección de recursos de información, clasificados y ordenados, cuyo
acceso al documento es siempre físico. Los recursos de información de dicha biblioteca están fijados, por regla general,
en papel y, salvo excepciones en bibliotecas especializadas, suelen componerse sus colecciones de volúmenes encuadernados
(monografías en su mayor parte) y publicaciones periódicas.\\
Limitaciones de una biblioteca tradicional:
\begin{enumerate}
\item El usuario tiene acceso de un tiempo limitado al recurso.
\item En muchos casos es que no contiene del mismo recurso para varios usuarios al mismo tiempo.
\end{enumerate} 

\section{Objetivo General}
Con la difusión de {\bf BABEL} se quiere que el papel como soporte de información esté compartiendo su hegemonía con los
soportes electrónicos y ópticos (como los libros digitales, la documentación accesible a través de la red, las bases de
datos en línea, etc.)\\
Ya no sera necesario desplazarse hasta la biblioteca o centro de documentación para conseguir la información que
necesitamos, se puede acceder al contenido de esa información almacenada en soporte digital (a distancia) y obtenerla al
momento.\\
Con {\bf BABEL} cambiaría la idea del acceso al documento físico por la idea de acceso al contenido. Cuyo propósito es
almacenar y conservar el máximo número de elementos de información.
\section{Objetivos Específicos}

\section{Recursos}

\section{Herramientas}
Si bien, el fin que nos justifica va mas allá de lo que tecnologicamente sería relevante, deberemos mencionar algo del
software que ayuda a nuestros propositos, cabe recalcar encarecidamente que lo que intentamos hacer aqui, va mas a allá
del software usado, construido, o necesario, nunca hay que olvidar la meta oficial.

\subsection{Herramientas para desarrollo}
En el cuadro~\ref{herramientas_desarrollo} se detallan las herramientas utilizadas en el desarrollo de nuestra herramienta.

\begin{table}
\begin{tabular}{l|l}
Herramienta & Descripción \\
\hline
git & Manejo de control de versiones. \\
& Asociada al sitio github.com (repositorio de software publico). \\
netbeans & IDE de desarrollo. \\

\end{tabular}
\caption{Herramientas que apoyan al desarrollo.}
\label{herramientas_desarrollo}
\end{table}

\subsection{Herramientas para despliegue}
En el cuadro~\ref{herramientas_despliegue} se detallan las herramientas utilizadas e imprescindibles para la instalación
y uso del software construido; ademas de ser estas, en las que se hizo la evaluación del desempeño adecuado.

\begin{table}
\begin{tabular}{l|r|l}
Herramienta    & Versión & Descripción                                  \\
\hline
Apache         &     2.0 & Servidor de aplicaciones web.                \\
MySQL          &     5.0 & Servidor de base de datos.                   \\
Cron           &     3.0 & Demonio para tareas automaticas.             \\
               &         & Necesario para tareas de mantenimiento e     \\
               &         & indexado.                                    \\
PHP            &     5.3 & Lenguage de programación.                    \\
ImageMagick    &     6.5 & Libreria para manipulación de imagenes.      \\
jQuery         &     1.6 & Libreria para desarrollo en JavaScript.      \\
Zend Framework &    1.11 & Conjunto de librerias para aplicaciones web. \\
\end{tabular}
\caption{Herramientas necesarias para la ejecución.}
\label{herramientas_despliegue}
\end{table}

En el cuadro~\ref{herramientas_zend} se muestran los paquetes de la libreria zend utilizados.

\begin{table}
\begin{tabular}{l|l}
Paquete & Descripción \\
\hline
Zend\_Application     & Librería encargada del arranque y control de    \\
					  & rutas en la aplicación. \\
Zend\_Auth            & Sistema encargado de la autentificación en la   \\
					  & aplicación. \\
Zend\_Controller      & Librería encargada del control del flujo de     \\
				      & informacion. \\
Zend\_Db              & Librería de acceso a base de datos. \\
Zend\_Form            & Generador de formularios con filtraje y         \\
					  & validación de sus contenidos. \\
Zend\_Search\_Lucene  & Motor de búsqueda. \\
Zend\_Translate       & Método para internacionalización (i18n) de la   \\
				      & aplicación. \\
Zend\_View            & Sistema de plantillas. \\
\end{tabular}
\caption{Paquetes de Zend utilizados.}
\label{herramientas_zend}
\end{table}

\section{Metodo o Técnica}
Inicialmente se utilizó un metodo clasico de desarrollo incremental, desarrollando la base que satisfaga las funciones
mas basicas, y de ahi ir especializando cada vez mas estas funciones.

\subsection{Recursos definidos}
Primeramente se establecieron recursos, sobre los que realizar funciones en el sistema, estos recursos se describen en el
cuadro~\ref{recursos_actuales}.

\begin{table}
\begin{tabular}{l|l}
Recurso & Descripción \\
\hline
Usuario     & Representación de una persona real, que tiene asignada    \\
            & multiples funciones que puede realizar en el sistema.     \\
Libro       & Representación de un archivo digital a ser compartido.    \\
Publicación & Representación de un \emph{libro} que ya posee meta-información. \\
Catalogo    & Agrupación de libros que comparten algun criterio subjetivo \\
		    & en comun.                                                 \\
Etiqueta    & Palabra clave que constituye un criterio de agrupacion.   \\
\end{tabular}
\caption{Recursos establecidos para el sistema.}
\label{recursos_actuales}
\end{table}

\subsection{Funciones establecidas}
Las funciones que ya se tienen, y los modulos encargados, se detallan en el cuadro~\ref{funciones_actuales}.

\begin{table}
\begin{tabular}{l|l}
Módulo & Función \\
\hline
Users    & Crear \emph{usuarios} del sistema.                           \\
	     & Mostrar los \emph{usuarios} actuales del sistema. 			\\
Auth     & Permitir a \emph{usuarios} del sistema autentificarse.       \\
Settings & Permitir a \emph{usuarios} cambiar algunas de sus propiedades \\
	     & establecidas. \\ 
Search   & Buscar \emph{libros} según algún criterio. 					\\
Catalogs & Navegar entre los \emph{catalogos}.		 					\\
	     & Crear \emph{catalogos} según algun criterio de agrupación.   \\
	     & Crear \emph{sub-catalogos} según algun criterio de especialización. \\
Tags     & Ver las \emph{etiquetas} creadas en el sistema. 				\\
Books    & Explorar entre colecciones de \emph{libros}. 				\\
	     & Generar miniaturas de un \emph{libro} especifico.			\\
	     & Clasificar un \emph{libro} dentro de un \emph{catalogo}.     \\
	     & Descargar un \emph{libro}.									\\
	     & Publicar un \emph{libro}.                                    \\
	     & Encontrar \emph{publicaciones} que no respecten a un \emph{libro}. \\
	     & Exportar toda la meta-información que se posee sobre las     \\
	     & \emph{publicaciones}. \\
	     & Importar meta-información de cualquier fuente externa.       \\
\end{tabular}
\caption{Funciones cubiertas por el sistema.}
\label{funciones_actuales}
\end{table}

\section{Justificación}

\end{document}          
