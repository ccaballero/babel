\documentclass[letter,11pt,oneside,spanish]{article}

\usepackage[utf8x]{inputenc}
\usepackage[spanish]{babel}
\usepackage{graphicx}
\usepackage{float}
\usepackage{anysize}

\usepackage[
pdfauthor={Carlos Caballero},%
pdftitle={Babel},%
colorlinks,%
citecolor=black,%
filecolor=black,%
linkcolor=black,%
%urlcolor=black
pdftex]{hyperref}

\marginsize{2.5cm}{2.5cm}{2.5cm}{2.5cm}
\renewcommand{\baselinestretch}{1.5}

%opening
\title{\textbf{Babel: Servidor de libros digitales}}
\author{Carlos Caballero\\
        Documento de Investigaci\'on\\
        Universidad Mayor de San Sim\'on\\
	\{cijkb.j@gmail.com\\}
\date{}
\begin{document}

\begin{titlepage}
\thispagestyle{empty}
\begin{center}
\large{\textsc{\bf Universidad Mayor De San Simón}}\\
\large{\textsc{\bf Facultad De Ciencias y Tecnología}}\\
\large{\textsc{\bf Sociedad Cientifica de Informática y Sistemas}}\\
\vspace{4.0cm}
\large{\bf Reducción de las brechas de accesibilidad a partir\\
del intercambio libre de recursos de aprendizaje}\\
\vspace{1.0cm}
\small{Carlos Eduardo Caballero Burgoa}
~\\
\small{\today}
\end{center}
\end{titlepage}

\newpage
\tableofcontents

\newpage
\section{Introducción}
El proposito de este documento es resumir todo el proceso de investigación
llevado a cabo para la construcción de una solución factible al problema de
intercambio de información entre estudiantes. A partir del desarrollo de un
sistema web al cual denomino 'Babel'.

Babel es una de las piezas de software desarrolladas en la sociedad científica
de sistemas e informática; esta intenta reducir las brechas de accesibilidad que
se perciben entre la comunidad estudiantil. En esencia consiste en un sitio web
desarrollado en el lenguaje de programación php, donde los usuarios pueden
compartir, ordenar, clasificar, y catalogar archivos en formato pdf.

Babel ha sido concebida con un lógica descentralizada de intercambio, es
decir, esta diseñado pensando en crear conexiones con otras instancias, ya sean
publicas o privadas, de modo que el rango de búsqueda pueda propagarse a una
variedad aún mayor que la de una única instancia.

\section{Antecedentes}
Debido a que no contamos con una biblioteca virtual en los predios de la universidad, surgió la
idea de realizar un proyecto que trate de crear un centro de información electrónico que 
proporcione documentos a sus usuarios.

Si bien los problemas de accesibilidad a los recursos de información se han visto reducidos por el
cada vez mayor acceso a Internet, aun el problema esta vigente, dadas esas condiciones se intento
subsanar algunos aspectos de este problema. Después de algunas investigaciones, se concluyo que el conjunto de dificultades que tienen las personas, pueden ser clasificados en cuatro tipos:

\begin{description}
\item [Problemas de accesibilidad] Referentes a los problemas de acceso al recurso necesario.
\item [Problemas de conocimiento] Referentes a los problemas de desconocimiento acerca del área 
sobre el que uno quiere desempeñarse.
\item [Problemas de tiempo] Referentes a los problemas en los que el tiempo requerido para alcanzar
el objetivo no puede ser cubierto.
\item [Problemas de compromiso] Referentes a los problemas en los que la intención por hacer lo
requerido es la carencia principal de las personas.
\end{description}

Entendiendo el conjunto de problemas, justificamos la creación del proyecto babel, en minimizar los
problemas de accesibilidad, siendo esta la primera barrera que se tiene al intentar alcanzar los
objetivos deseados.

\section{Justificación}
Ya no sera necesario desplazarse hasta la biblioteca o centro de documentación para conseguir la
información que necesitamos, se puede acceder al contenido de esa información almacenada en soporte
digital (a distancia) y obtenerla al momento.

Con babel cambiaría la idea del acceso al documento físico por la idea de acceso al contenido. Cuyo propósito es almacenar y conservar el máximo número de elementos de información.

Con la difusión de babel se quiere que el papel como soporte de información esté compartiendo su
hegemonía con los soportes electrónicos y ópticos (como los libros digitales, la documentación
accesible a través de la red, las bases de datos en línea, etc.)

\section{Planteamiento del Problema}

\section{Formulación del Problema}

\section{Objetivo General}
Minimizar los problemas de accesibilidad al conocimiento para mejorar los niveles de rendimiento
académico de los estudiantes de la universidad.

\section{Objetivos Específicos}
\begin{itemize}
\item Fomentar la cultura de participación y colaboración entre las personas de nuestro medio.
\item Ampliar los canales de intercambio de recursos entre las personas.
\item Ganar experiencia en el desarrollo de proyectos de software libre.
\end{itemize}

\section{Hipótesis}
\section{Aporte científico}
\section{Diseño metodológico y teórico}
\section{Desarrollo del proyecto}
\section{Conclusiones y recomendaciones}

\newpage
\begin{thebibliography}{99}
 \bibitem{Conetivismo} \textsc{George Siemens.:}
              \textit{\textbf{Conectivismo: Una teor\'ia de aprendizaje para la era digital.}}
              \par Diciembre 12, 2004.

 \bibitem{tdl} \textsc{George Siemens.:}
            \textit{\textbf{The value of diversity in learning.}}
            \par \url{http://www.connectivism.ca/?p=24}

 \bibitem{as} \textsc{Andres Schuschny.:}
             \textit{\textbf{Conectivismo: una teor\'ia del aprendizaje para la era digital.}}
             \par \url{http://humanismoyconectividad.wordpress.com/2009/01/14/conectivismo-siemens/}

 \bibitem{social} \textsc{\textbf{The Social Learning Theory of Julian B. Rotter}}
                 \par \url{http://psych.fullerton.edu/jmearns/rotter.htm}
 
 \bibitem{diapositiva} \textsc{Fernando Luna Pizarro Quinteros.:}
                       \textsc{\textbf{Estrategias de aprendizaje 2.0 - enfoque te\'orico}}

 \bibitem{web} \textsc{James Governor, Dion HinchclifFe, Duane Nickull.:}
               \textsc{\textbf{ Web 2.0 Architectures.}}
               \par O'Reilly
                                       
 \bibitem{web2.0} \textsc{Crist\'obal Cobo Roman\'i, Hugo Pardo Kuklinski.:}
                  \textsc{\textbf{Planeta Web 2.0 inteligencia colectiva o medio fast food}}
                  \par \url{http://www.planetaweb2.net/}                   
  
\end{thebibliography}
\end{document}
