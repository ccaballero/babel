\documentclass[letter,11pt,twoside,spanish]{article}

\usepackage[utf8x]{inputenc}
\usepackage[spanish]{babel}
\usepackage{graphicx}
\usepackage{float}
\usepackage{anysize}
\usepackage{multicol}

\usepackage[
pdfauthor={Carlos Caballero},%
pdftitle={Babel},%
colorlinks,%
citecolor=black,%
filecolor=black,%
linkcolor=black,%
%urlcolor=black
pdftex]{hyperref}

\marginsize{2.5cm}{2.5cm}{2.5cm}{2.5cm}
\renewcommand{\baselinestretch}{1.5}

%opening
\title{Reducción de las brechas de acceso a la información a partir del
intercambio libre de recursos de aprendizaje}
\date{19 de Agosto del 2013}
\author{\emph{Carlos Caballero}\\
Documento de Investigación\\
Universidad Mayor de San Simón\\
cijkb.j at gmail.com\\}
\begin{document}
\begin{multicols}{2}
\maketitle
\pagestyle{empty}
Se hablará acerca del proceso de investigación llevado a cabo para la
construcción de una solución factible al problema del intercambio de
información entre personas. A partir del desarrollo de un sistema web
al que me referiré como: 'Babel'.

\emph{Babel} es una de las piezas de software desarrolladas en la sociedad científica
de sistemas e informática (UMSS); que esta encaminada a reducir las brechas de
acceso a la información que se han percibido entre la comunidad estudiantil.
Esencialmente consiste en un sitio web desarrollado en el lenguaje de
programación PHP, donde los usuarios pueden compartir, ordenar, clasificar, y
catalogar archivos en formato PDF.

Esta solución ha sido concebida con un lógica descentralizada de intercambio, es
decir, esta diseñada para crear multiples conexiones con otras instancias, ya
sean publicas o privadas, de modo que el rango de búsqueda pueda propagarse a
una variedad aún mayor que la de una única instancia (P2P).

Se ha estructurado el proyecto en cuatro grandes tareas:

\begin{itemize}
\item Busqueda de documentos
\item Intercambio de documentos
\item Clasificación de documentos
\item Valoración de documentos
\end{itemize}

Despues de muchos meses de desarrollo y refactorización de las funciones, se
obtuvo una versión estable del sistema. Este se encuentra alojado actualmente
en el sitio \url{http://babel.scesi.org}, y se ha publicado el codigo fuente del
mismo en el sitio \url{https://github.com/ccaballero/babel} para la instalación
libre de otras instancias descentralizadas.
\end{multicols}
\end{document}
