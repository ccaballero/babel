\documentclass[letter,12pt]{article}

\usepackage[T1]{fontenc}
\usepackage{lmodern}
\usepackage{textcomp}
\renewcommand*\familydefault{\sfdefault}

\usepackage[spanish]{babel}
\usepackage[utf8x]{inputenc}

\usepackage[pdftex]{graphicx}
\usepackage{pifont}
\usepackage[
pdfauthor={Carlos Caballero - Ubaldino Zurita},%
pdftitle={Proyecto Babel},%
colorlinks,%
citecolor=black,%
filecolor=black,%
linkcolor=black,%
%urlcolor=black
pdftex]{hyperref}

\usepackage{fancyhdr}
\usepackage{lastpage}
\pagestyle{fancy}

% Para la primera página
\fancypagestyle{plain}{
\fancyhead[l]{}
\fancyhead[r]{}
\fancyhead[c]{}
\renewcommand{\headrulewidth}{0.5pt}
\fancyfoot[l]{SCESI \\ Sociedad Científica de Estudiantes de Sistemas e Informática}
\fancyfoot[c]{}
\fancyfoot[r]{\thepage/\pageref{LastPage}}
\renewcommand{\footrulewidth}{0.5pt}}

% Para el resto de páginas
\lhead{Proyecto Babel}
\chead{}
\rhead{\includegraphics[width=0.1\textwidth]{../img/scesi.png}}
\renewcommand{\headrulewidth}{0.4pt}
\lfoot{SCESI \\ Sociedad Científica de Estudiantes de Sistemas e Informática \\ 
\url {http://scesi.fcyt.umss.edu.bo}}
\cfoot{}
\rfoot{\thepage/\pageref{LastPage}}
\renewcommand{\footrulewidth}{0.4pt}

\title{\bf Babel}
\author{Carlos Caballero\\Ubaldino Zurita} 

\begin{document}
\maketitle
\begin{center}\includegraphics[width=0.48\textwidth]{../img/babel.png}\end{center}
\begin{center}\url {http://scesi.fcyt.umss.edu.bo}\end{center}
\pagebreak

\tableofcontents
\pagebreak

\section{Introducción}
Babel es una de las piezas de software desarrolladas en la sociedad científica de sistemas e
informática; esta intenta reducir las brechas de accesibilidad que se perciben entre la comunidad
estudiantil. En esencia consiste en un sitio web desarrollado en el lenguaje de programación php,
donde los usuarios pueden compartir, ordenar, clasificar, y catalogar archivos en formato pdf.

Babel ha sido concebida con un lógica descentralizada de intercambio, es decir, esta diseñado
pensando en crear conexiones con otras instancias, ya sea publicas o privadas, de modo que el rango
de búsqueda pueda propagarse a una variedad aún mayor que la de una única instancia.

Babel ademas es software libre, lo que implica que cualquier usuario puede instalarse una instancia
en su computador y conectarse con otras instancias accesibles vía web.

\section{Antecedentes}
Debido a que no contamos con una biblioteca virtual en los predios de la universidad, surgió la
idea de realizar un proyecto que trate de crear un centro de información electrónico que 
proporcione documentos a sus usuarios.

Si bien los problemas de accesibilidad a los recursos de información se han visto reducidos por el
cada vez mayor acceso a Internet, aun el problema esta vigente, dadas esas condiciones se intento
subsanar algunos aspectos de este problema. Después de algunas investigaciones, se concluyo que el conjunto de dificultades que tienen las personas, pueden ser clasificados en cuatro tipos:

\begin{description}
\item [Problemas de accesibilidad] Referentes a los problemas de acceso al recurso necesario.
\item [Problemas de conocimiento] Referentes a los problemas de desconocimiento acerca del área 
sobre el que uno quiere desempeñarse.
\item [Problemas de tiempo] Referentes a los problemas en los que el tiempo requerido para alcanzar
el objetivo no puede ser cubierto.
\item [Problemas de compromiso] Referentes a los problemas en los que la intención por hacer lo
requerido es la carencia principal de las personas.
\end{description}

Entendiendo el conjunto de problemas, justificamos la creación del proyecto babel, en minimizar los
problemas de accesibilidad, siendo esta la primera barrera que se tiene al intentar alcanzar los
objetivos deseados.

Es así que el 14 de junio del 2011, se obtiene el primer prototipo del sistema informático; después
de varias correcciones y cambios se llega a una version estable el 27 de julio del 2011, de ahí en
adelante se siguen desarrollando hasta la actualidad, ya sean correcciones urgentes, nuevas
funcionalidades y mejoras de compatibilidad.

\section{Definición del Problema}
La biblioteca tradicional se entiende como una colección de recursos de información, clasificados
y ordenados, cuyo acceso al documento es siempre físico. Los recursos de información de dicha
biblioteca están fijados, por regla general, en papel y, salvo excepciones en bibliotecas
especializadas, suelen componerse sus colecciones de volúmenes encuadernados (monografías en su
mayor parte) y publicaciones periódicas.

Limitaciones de una biblioteca tradicional:

\begin{enumerate}
\item El usuario tiene acceso al recurso por un tiempo limitado.
\item En la mayoría de los casos existe un único recurso debe ser compartido por varios usuarios.
\end{enumerate} 

\section{Objetivo General}
Minimizar los problemas de accesibilidad al conocimiento para mejorar los niveles de rendimiento
académico de los estudiantes de la universidad.

\section{Objetivos Específicos}
\begin{itemize}
\item Fomentar la cultura de participación y colaboración entre las personas de nuestro medio.
\item Ampliar los canales de intercambio de recursos entre las personas.
\item Ganar experiencia en el desarrollo de proyectos de software libre.
\end{itemize}

\section{Herramientas}
Si bien, el fin que nos justifica va mas allá de lo que tecnológicamente sería relevante, deberemos
mencionar algo del software que ayuda a nuestros propósitos.

\subsection{Herramientas para desarrollo}
En el cuadro~\ref{herramientas_desarrollo} se detallan las herramientas utilizadas en el desarrollo
de nuestra herramienta.

\begin{table}
\begin{tabular}{l|l}
Herramienta & Descripción                                                  \\
\hline
git         & Manejo de control de versiones.                                \\
            & Asociada al sitio github.com (repositorio de software publico). \\
netbeans    & IDE de desarrollo.                                               \\
\end{tabular}
\caption{Herramientas que apoyan al desarrollo.}
\label{herramientas_desarrollo}
\end{table}

\subsection{Herramientas para despliegue}
En el cuadro~\ref{herramientas_despliegue} se detallan las herramientas utilizadas e
imprescindibles para la instalación y uso del software construido; ademas de ser estas, en las
que se hizo la evaluación del desempeño adecuado.

\begin{table}
\begin{tabular}{l|r|l}
Herramienta    & Versión & Descripción                            \\
\hline
Apache         &     2.2 & Servidor de aplicaciones web.            \\
vsFTPd         &   2.3.5 & Servidor FTP (transferencia de ficheros). \\
MySQL          &     5.1 & Servidor de base de datos.                 \\
pam\_MySQL     &     0.7 & Modulo de autentificación para MySQL.       \\
Cron           &       - & Demonio para tareas automáticas.             \\
               &         & Necesario para tareas de mantenimiento e      \\
               &         & indexado.                                      \\
PHP            &     5.3 & Lenguaje de programación.                       \\
ImageMagick    &     6.5 & Liberia para manipulación de imágenes.           \\
jQuery         &     1.6 & Liberia para desarrollo en JavaScript.            \\
Zend Framework &    1.11 & Conjunto de librerías para aplicaciones web.       \\
\end{tabular}
\caption{Herramientas necesarias para la ejecución.}
\label{herramientas_despliegue}
\end{table}

En el cuadro~\ref{herramientas_zend} se muestran los paquetes de la libreria zend utilizados.

\begin{table}
\begin{tabular}{l|l}
Paquete               & Descripción                                \\
\hline
Zend\_Application     & Librería encargada del arranque y control de \\
                      & rutas en la aplicación.                       \\
Zend\_Auth            & Sistema encargado de la autentificación en la  \\
                      & aplicación.                                     \\
Zend\_Controller      & Librería encargada del control del flujo de      \\
                      & información.                                      \\
Zend\_Db              & Librería de acceso a base de datos.                \\
Zend\_Form            & Generador de formularios con filtraje y             \\
                      & validación de sus contenidos.                        \\
Zend\_Search\_Lucene  & Motor de búsqueda.                                    \\
Zend\_Translate       & Método para internacionalización (i18n) de la          \\
                      & aplicación.                                             \\
Zend\_View            & Sistema de plantillas.                                   \\
\end{tabular}
\caption{Paquetes de Zend utilizados.}
\label{herramientas_zend}
\end{table}

\section{Método o Técnica}
Inicialmente se utilizó un método clásico de desarrollo incremental, de\-sa\-rro\-llan\-do la
base que satisfaga las funciones mas básicas, de ahí partir en dos caminos; ya sea especializando
cada vez mas las funciones existentes, y por el otro lado implementar nuevas funcionalidades.

\subsection{Recursos definidos}
Primeramente se establecieron recursos, sobre los que realizar funciones en el sistema, estos
recursos se describen en el cuadro~\ref{recursos_actuales}.

\begin{table}
\begin{tabular}{l|l}
Recurso     & Descripción                                                 \\
\hline
Usuario     & Representación de una persona real, que tiene asignada        \\
            & múltiples funciones que puede realizar en el sistema.          \\
Libro       & Representación de un archivo digital a ser compartido.          \\
Publicación & Representación de un \emph{libro} que ya posee meta-información. \\
Catalogo    & Agrupación de libros que comparten algún criterio subjetivo       \\
            & en común.                                                          \\
Etiqueta    & Palabra clave que constituye un criterio de agrupación.             \\
\end{tabular}
\caption{Recursos establecidos para el sistema.}
\label{recursos_actuales}
\end{table}

\subsection{Funciones establecidas}
Las funciones que ya se tienen, y los módulos encargados, se detallan en el 
cuadro~\ref{funciones_actuales}.

\begin{table}
\begin{tabular}{l|l}
Módulo   & Función                                                     \\
\hline
Users    & - Crear \emph{usuarios} del sistema.                          \\
         & - Mostrar los \emph{usuarios} actuales del sistema.            \\
Auth     & - Permitir a \emph{usuarios} del sistema autentificarse.        \\
Settings & - Permitir a \emph{usuarios} cambiar algunas de sus propiedades  \\
         & establecidas.                                                     \\ 
Search   & - Buscar \emph{libros} según algún criterio.                       \\
Catalogs & - Navegar entre los \emph{catalogos}.                               \\
         & - Crear \emph{catalogos} según algún criterio de agrupación.         \\
         & - Crear \emph{sub-catalogos} según algún criterio de especialización. \\
Tags     & - Ver las \emph{etiquetas} creadas en el sistema.                      \\
Books    & - Explorar entre colecciones de \emph{libros}.                          \\
         & - Generar miniaturas de un \emph{libro} especifico.                      \\
         & - Clasificar un \emph{libro} dentro de un \emph{catalogo}.                \\
         & - Descargar un \emph{libro}.                                               \\
         & - Publicar un \emph{libro}.                                                 \\
         & - Encontrar \emph{publicaciones} que respecten a un \emph{libro}.            \\
         & - Exportar toda la meta-información que se posee sobre las                    \\
         & \emph{publicaciones}.                                                          \\
         & - Importar meta-información de cualquier fuente externa.                        \\
FTP      & - Ver la colección de \emph{libros} desde una servidor FTP.                      \\
         & - Compartir \emph{libros} con la aplicación por medio de un servidor              \\
         & FTP autentificado.                                                                 \\
\end{tabular}
\caption{Funciones cubiertas por el sistema.}
\label{funciones_actuales}
\end{table}

\section{Justificación}
Ya no sera necesario desplazarse hasta la biblioteca o centro de documentación para conseguir la
información que necesitamos, se puede acceder al contenido de esa información almacenada en soporte
digital (a distancia) y obtenerla al momento.

Con babel cambiaría la idea del acceso al documento físico por la idea de acceso al contenido. Cuyo propósito es almacenar y conservar el máximo número de elementos de información.

Con la difusión de babel se quiere que el papel como soporte de información esté compartiendo su
hegemonía con los soportes electrónicos y ópticos (como los libros digitales, la documentación
accesible a través de la red, las bases de datos en línea, etc.)

\end{document}          
